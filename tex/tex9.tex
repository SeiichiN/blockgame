\documentclass[dvipdfmx]{jsarticle}

\title{ブロック落しゲーム(JavaScript)}
\author{Seiichi Nukayama}
\date{2020-06-21}
\usepackage{tcolorbox}
\usepackage{color}
\usepackage{listings, plistings}

% Java
\lstset{% 
  frame=single,
  backgroundcolor={\color[gray]{.9}},
  stringstyle={\ttfamily \color[rgb]{0,0,1}},
  commentstyle={\itshape \color[cmyk]{1,0,1,0}},
  identifierstyle={\ttfamily}, 
  keywordstyle={\ttfamily \color[cmyk]{0,1,0,0}},
  basicstyle={\ttfamily},
  breaklines=true,
  xleftmargin=0zw,
  xrightmargin=0zw,
  framerule=.2pt,
  columns=[l]{fullflexible},
  numbers=left,
  stepnumber=1,
  numberstyle={\scriptsize},
  numbersep=1em,
  language={Java},
  lineskip=-0.5zw,
  morecomment={[s][{\color[cmyk]{1,0,0,0}}]{/**}{*/}},
}
%\usepackage[dvipdfmx]{graphicx}
\usepackage{url}
\usepackage[dvipdfmx]{hyperref}
\usepackage{amsmath, amssymb}
\usepackage{itembkbx}
\usepackage{eclbkbox}	% required for `\breakbox' (yatex added)
\usepackage{setspace}
\usepackage{multicol}
\fboxrule=1pt
\parindent=1em
\begin{document}

%% 修正時刻: Sun Jun 21 08:35:35 2020


\section{タイマーでユーザーを忙せる}

まず、200行目あたりの let でいくつかの変数を定義しているところに、以下の変数定義を
書き加えてください。

\begin{lstlisting}
 let gameOn;        // ゲームが実行中である
 let jikan;         // タイマー
\end{lstlisting}

gamehaishi()関数の上に以下の関数を書き加えてください。

\begin{lstlisting}
 /**
 * ユーザーが何もしなかったら、自動的に下向き矢印キーが
 * 押されたと同じことになる。しかも jikan はだんだんと
 * 短くなる。50ミリ秒以下になると1000ミリ秒に戻る。
 */
function jikandeugokasu() {
  if (gameOn) {
    // 自動で下向きキーが押されるのと同じ。
    const ev = { keyCode: 40 };
    ugokasu(ev);

    jikan = jikan - 2;
    if (jikan < 50) {
      jikan = 1000;
    }
    document.getElementById('jikan').textContent = jikan;
    setTimeout(jikandeugokasu, jikan);
  }
}
\end{lstlisting}

そして、gamekaishi()関数の cg.clearRect(0, 0, 239, 439) の後に以下の記述を
書き加えてください。

\begin{lstlisting}
   // 画面を消す
  cg.clearRect(0, 0, 239, 439);

  // 得点をゼロにする
  tensu = 0;

  // ゲーム実行中にする
  gameOn = true;

  // タイマーを1秒にセット
  jikan = 1000;

  // 1秒ごとに jikandeugoku() を動かす。
  setTimeout( jikandeugokasu, jikan );
\end{lstlisting}

そして、index.html の \verb!<footer>!の中に以下の記述を加えてください。

\begin{lstlisting}
  <footer>
    <small>&copy; 2019 Seiichi Nukayama</small>
    <div class="timer"><span id="jikan"></span>/1000 秒</div>    <!-- <== -->
  </footer>
\end{lstlisting}

これでブラウザで動かしてみると、ゲーム開始以後は、何もしなくても勝手にブロックが
落ちてきます。

タイマーがフッターあたりに表示されています。

\subsection{ゲームオーバー処理}

ゲームオーバーの判定は、次のブロックを置く処理をしているところで行います。
以下のように changeToNextBlock()関数に追加してください。

\begin{lstlisting}
function changeToNextBlock() {
  ... (省略) ...
  muki = nextBlock.muki;

  // 新しいブロックが置けるかどうかを確認して、
  // もし置けなかったら、ゲームオーバー
  if (! kakunin(col, row, syurui, muki)) {
    otoGameover();
    alert('ゲームオーバー');
    gameOn = false;
  }
}
\end{lstlisting}

また、以下のように音を鳴らす関数も追加しておいてください。

\begin{lstlisting}
function otoGameover() {
    document.getElementById('gameover').play();
}
function otoKieru() {
    document.getElementById('kieru').play();
}
function otoZenbu() {
    document.getElementById('zenbu').play();
} 
\end{lstlisting}

そして、得点処理のところに音を追加します。

\begin{lstlisting}
 /**
 * 得点処理
 * そろっている行が多いほど得点が上がる。
 */
function tokuten(num) {
  switch (num) {
  case 1:
    tensu = tensu + 10;
    otoKieru();                    // <===
    break;
  case 2:
    tensu = tensu + 50;
    otoKieru();                    // <===
    break;
  case 3:
    tensu = tensu + 100;
    otoKieru();                    // <===
    break;
  case 4:
    tensu = tensu + 500;
    otoZenbu();                    // <===
    break;
  }
  document.getElementById('tokuten').textContent = tensu;
}
\end{lstlisting}


以上で完成です。

\include{end}

%% 修正時刻: Mon Jun 22 19:11:16 2020
