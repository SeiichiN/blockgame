\documentclass[dvipdfmx]{jsarticle}

\title{ブロック落しゲーム(JavaScript)}
\author{Seiichi Nukayama}
\date{2020-06-21}
\usepackage{tcolorbox}
\usepackage{color}
\usepackage{listings, plistings}

% Java
\lstset{% 
  frame=single,
  backgroundcolor={\color[gray]{.9}},
  stringstyle={\ttfamily \color[rgb]{0,0,1}},
  commentstyle={\itshape \color[cmyk]{1,0,1,0}},
  identifierstyle={\ttfamily}, 
  keywordstyle={\ttfamily \color[cmyk]{0,1,0,0}},
  basicstyle={\ttfamily},
  breaklines=true,
  xleftmargin=0zw,
  xrightmargin=0zw,
  framerule=.2pt,
  columns=[l]{fullflexible},
  numbers=left,
  stepnumber=1,
  numberstyle={\scriptsize},
  numbersep=1em,
  language={Java},
  lineskip=-0.5zw,
  morecomment={[s][{\color[cmyk]{1,0,0,0}}]{/**}{*/}},
}
%\usepackage[dvipdfmx]{graphicx}
\usepackage{url}
\usepackage[dvipdfmx]{hyperref}
\usepackage{amsmath, amssymb}
\usepackage{itembkbx}
\usepackage{eclbkbox}	% required for `\breakbox' (yatex added)
\usepackage{setspace}
\usepackage{multicol}
\fboxrule=1pt
\parindent=1em
\begin{document}

%% 修正時刻: Sun Jun 21 08:35:35 2020


\section{横1列がそろっているかをチェックして得点を計算する}

矢印キーの下向きで底に着いたとき、もしブロックが横1列そろっていたら、
得点計算して、そろっているブロックを削除します。2列〜4列そろっていたら、
その得点を計算して削除します。

横1列そろっているたらの関数名を checkHorizontalLine() とします。
それを ugokasu()関数の下向き矢印をチェックしている \verb!(case 40:)! の atBottom()関数を呼び出しているところの下に入れます。

\begin{lstlisting}
 function ugokasu(e) {
  ... (省略) ...
   case 40:
    // 画面データの配列 joutai にブロックの種類を埋め込む
    atBottom(col, row, syurui, muki);
    // そろっている行があれば、得点、削除、再描画を行う。
    checkHorizontalLine();                        // <=== これ
    printJoutai();
\end{lstilsting}



\include{end}

%% 修正時刻: Mon Jun 22 09:41:25 2020
