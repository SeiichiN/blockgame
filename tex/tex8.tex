\documentclass[dvipdfmx]{jsarticle}

\title{ブロック落しゲーム(JavaScript)}
\author{Seiichi Nukayama}
\date{2020-06-21}
\usepackage{tcolorbox}
\usepackage{color}
\usepackage{listings, plistings}

% Java
\lstset{% 
  frame=single,
  backgroundcolor={\color[gray]{.9}},
  stringstyle={\ttfamily \color[rgb]{0,0,1}},
  commentstyle={\itshape \color[cmyk]{1,0,1,0}},
  identifierstyle={\ttfamily}, 
  keywordstyle={\ttfamily \color[cmyk]{0,1,0,0}},
  basicstyle={\ttfamily},
  breaklines=true,
  xleftmargin=0zw,
  xrightmargin=0zw,
  framerule=.2pt,
  columns=[l]{fullflexible},
  numbers=left,
  stepnumber=1,
  numberstyle={\scriptsize},
  numbersep=1em,
  language={Java},
  lineskip=-0.5zw,
  morecomment={[s][{\color[cmyk]{1,0,0,0}}]{/**}{*/}},
}
%\usepackage[dvipdfmx]{graphicx}
\usepackage{url}
\usepackage[dvipdfmx]{hyperref}
\usepackage{amsmath, amssymb}
\usepackage{itembkbx}
\usepackage{eclbkbox}	% required for `\breakbox' (yatex added)
\usepackage{setspace}
\usepackage{multicol}
\fboxrule=1pt
\parindent=1em
\begin{document}

%% 修正時刻: Sun Jun 21 08:35:35 2020


\section{横1列がそろっているかをチェックして得点を計算する}

矢印キーの下向きで底に着いたとき、もしブロックが横1列そろっていたら、
得点計算して、そろっているブロックを削除します。2列〜4列そろっていたら、
その得点を計算して削除します。

横1列そろっているたらの関数名を checkHorizontalLine() とします。
それを ugokasu()関数の下向き矢印をチェックしている \verb!(case 40:)! の atBottom()関数を呼び出しているところの下に入れます。

\begin{lstlisting}
 function ugokasu(e) {
  ... (省略) ...
   case 40:
    // 画面データの配列 joutai にブロックの種類を埋め込む
    atBottom(col, row, syurui, muki);
    // そろっている行があれば、得点、削除、再描画を行う。
    checkHorizontalLine();                        // <=== これ
    printJoutai();
\end{lstilsting}

次に以下の記述を kakunin()関数の上に書き加えてください。


\begin{lstlisting}
/**
 * 横1列そろったら行う処理
 * そろっているかどうかを確認する。
 * そろってたら、そろっている行数分、得点する。
 * そろっている行を削除して再描画。
 */
 function checkHorizontalLine() {
   const lineNo = [];   // 横一列そろっている行の配列
   let count;        // 100でないセルの数
   let x, y;

   // 1行ごとに100でないセルの数を数える
   for (y = 0; y < 21; y++) {
     count = 0;
     for (x = 1; x < 11; x ++) {
       if (joutai[y][x] !== 100) {
         count++;
       }
     }

     // もし、10個のセル全部が100でなかったら、
     if (count === 10) {
       lineNo.push(y);   // その行番号をlineNoに入れる。
     }
   }

   // lineNoには、そろっている行の番号が配列ではいっている。
   // 上の行から順番に、そろっている行を削除し、上の行を下につめる。
   if (lineNo.length > 0) {
     tokuten(lineNo.length);
     lineNo.forEach((ele) => {
       if (ele !== 0) {
         moveJoutai(ele);
       }
     });
     printJoutai();
     reRender();
   }
 } 
\end{lstlisting}

この関数の中では、次の関数を呼び出しています。

\begin{itemize}
 \item tokuten()関数 -- 得点計算をする。
 \item moveJoutai()関数 -- そろっている行を削除して、上の行をもってくる。
 \item printJoutai()関数 -- 画面データの状態を見る。(すでに作成ずみ)
 \item reRender()関数 -- 画面を再描画する。
\end{itemize}

次の記述を checkHorizontalLine()関数の上に書き加えます。

\begin{lstlisting}
/**
 * 得点処理
 * そろっている行が多いほど得点が上がる。
 */
 function tokuten(num) {
   switch (num) {
     case 1:
     tensu = tensu + 10;
     break;
     case 2:
     tensu = tensu + 50;
     break;
     case 3:
     tensu = tensu + 100;
     break;
     case 4:
     tensu = tensu + 500;
     break;
   }
   document.getElementById('tokuten').textContent = tensu;
 }
\end{lstlisting}

tensu という変数を使用していますので、それを作ります。\\
let muki; // ブロックの向き \\
と書かれているところのすぐ下に以下を記述してください。

\fbox{let tensu; // 得点}

次に moveJoutai()関数を書きます。
これを tokuten()関数の上に書き加えてください。

\begin{lstlisting}
 /**
 * そろっている行を削除して上の行をつめる
 * 引数:
 *   y -- 行の番号
 */
function moveJoutai(y) {
  let x;
  
  if (y !== 0) {
    for (x = 1; x < 11; x++) {
      // 一つ上のセルの内容をコピーする
      joutai[y][x] = joutai[y - 1][x];
    }
    // 再帰処理。
    // すなわち、一つ上の行にも同じことをする。
    // ただ、第0行はしない。そこでこの再帰は終わる。
    moveJoutai(y - 1);
  }
  // 第0行は100で埋める。
  // なぜなら第0行は新しく生み出された行だから。
  for (x = 1; x < 11; x++) {
    joutai[0][x] = 100;
  }
}
\end{lstlisting}

次に再描画を行う関数 reRender() を書きます。
moveJoutai()関数の上に書き加えてください。

\begin{lstlisting}
 /**
 * 再描画処理
 * 画面データ(joutai配列)をもとにして再描画する。
 */
 function reRender() {
   const cv = document.getElementById('game').getContext('2d');
   
   cv.clearRect(20, 0, 219, 419);
   // console.log('----------------------');
   for (y = 0; y < 21; y++) {
     for (x = 1; x < 11; x++) {
       if (joutai[y][x] !== 100) {
         // console.log('joutai y:' + y + ' x:' + x + ' j:' + joutai[y][x]);
         cv.fillStyle = bcolor[joutai[y][x]];
         cv.strokeStyle = '#333333';
         cv.lineWidth = 3;
         cv.fillRect(x * 20, y * 20, 20, 20);
         cv.strokeRect(x * 20, y * 20, 20, 20);
       }
     }
   }
 }
\end{lstlisting}

ブラウザで動かしてみます。
横1列がそろうと、その行が削除され、上の行が降りてきます。
複数行だと、複数行が削除されます。
得点も表示されます。







\end{document}

%% 修正時刻: Sat May  2 15:10:04 2020


%% 修正時刻: Mon Jun 22 13:38:33 2020
