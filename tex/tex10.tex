\documentclass[dvipdfmx]{jsarticle}

\title{ブロック落しゲーム(JavaScript)}
\author{Seiichi Nukayama}
\date{2020-06-21}
\usepackage{tcolorbox}
\usepackage{color}
\usepackage{listings, plistings}

% Java
\lstset{% 
  frame=single,
  backgroundcolor={\color[gray]{.9}},
  stringstyle={\ttfamily \color[rgb]{0,0,1}},
  commentstyle={\itshape \color[cmyk]{1,0,1,0}},
  identifierstyle={\ttfamily}, 
  keywordstyle={\ttfamily \color[cmyk]{0,1,0,0}},
  basicstyle={\ttfamily},
  breaklines=true,
  xleftmargin=0zw,
  xrightmargin=0zw,
  framerule=.2pt,
  columns=[l]{fullflexible},
  numbers=left,
  stepnumber=1,
  numberstyle={\scriptsize},
  numbersep=1em,
  language={Java},
  lineskip=-0.5zw,
  morecomment={[s][{\color[cmyk]{1,0,0,0}}]{/**}{*/}},
}
%\usepackage[dvipdfmx]{graphicx}
\usepackage{url}
\usepackage[dvipdfmx]{hyperref}
\usepackage{amsmath, amssymb}
\usepackage{itembkbx}
\usepackage{eclbkbox}	% required for `\breakbox' (yatex added)
\usepackage{setspace}
\usepackage{multicol}
\fboxrule=1pt
\parindent=1em
\begin{document}

%% 修正時刻: Sun Jun 21 08:35:35 2020


\section{全コード}

\begin{lstlisting}[caption=index.html]
 <!doctype html>
<html lang="ja">
    <head>
        <meta charset="utf-8" />
        <title>game</title>
        <link rel="stylesheet" href="style.css" />
        <script src="program.js"></script>
        
    </head>
	<body onload="hajime()" onkeydown="ugokasu(event)">
        <div id="playarea">
            <div class="header">
                <h1>落ちものパズル</h1>
                <div class="score">
                    スコア:<span id="tokuten">0</span>
                </div>
            </div>
            <canvas id="back" width="240" height="440"></canvas>
            <canvas id="game" width="240" height="440"></canvas>
            <canvas id="tsugi" width="80" height="80"></canvas>
            <button id="kaishibtn" onclick="gamekaishi()">ゲーム<br>スタート</button>
            <footer>
                <small>&copy; 2019 Seiichi Nukayama</small>
                <div class="timer"><span id="jikan"></span>/1000 秒</div>
            </footer>
        </div><!-- #playarea -->
        <div id="dataarea">
        </div>
        <audio id="kaiten" preload="auto" src="oto/kaiten.mp3"></audio>
        <audio id="ochiru" preload="auto" src="oto/ochiru.mp3"></audio>
        <audio id="don" preload="auto" src="oto/don.mp3"></audio>
        <audio id="kieru" preload="auto" src="oto/kieru.mp3"></audio>
        <audio id="zenbu" preload="auto" src="oto/zenbu.mp3"></audio>
        <audio id="gameover" preload="auto" src="oto/gameover.mp3"></audio>
    </body>
</html>
\end{lstlisting}

\begin{lstlisting}[caption=style.css]
 @charset "utf-8";

.header {
    position: absolute;
    left: 20px;
    top: 10px;
}
.score {
    width: 380px;
}
#back,
#game {
    position: absolute;
    left: 20px;
    top: 150px;
}
#back {
    background-color: #000;
}
#game {
    background-color: transparent;
}
#tsugi {
    position: absolute;
    left: 300px;
    top: 150px;
    background-color: #000;
}
#kaishibtn {
    position: absolute;
    left: 300px;
    top: 300px;
    width: 80px;
    height: 50px;
    padding: 0;
    font-size: 0.8em;
}
footer {
    position: absolute;
    top: 600px;
    left: 20px;
}

#dataarea {
    position: absolute;
    left: 500px;
    top: 100px;
}

#dataarea td {
    border: solid 1px #aaa;
    width: 20px;
    height: 20px;
}

#dataarea table {
    border-collapse: collapse;
}
\end{lstlisting}

\begin{lstlisting}[caption=program.js]
const block = [
  [      // block0
    [
      [0, 0, 0, 0],
      [1, 1, 1, 0],
      [0, 1, 0, 0],
      [0, 0, 0, 0]
    ],
    [
      [0, 1, 0, 0],
      [0, 1, 1, 0],
      [0, 1, 0, 0],
      [0, 0, 0, 0]
    ],
    [
      [0, 1, 0, 0],
      [1, 1, 1, 0],
      [0, 0, 0, 0],
      [0, 0, 0, 0]
    ],
    [
      [0, 1, 0, 0],
      [1, 1, 0, 0],
      [0, 1, 0, 0],
      [0, 0, 0, 0]
    ],
  ],
  [      // block1
    [
      [0, 0, 0, 0],
      [1, 1, 1, 0],
      [1, 0, 0, 0],
      [0, 0, 0, 0]
    ],
    [
      [1, 0, 0, 0],
      [1, 0, 0, 0],
      [1, 1, 0, 0],
      [0, 0, 0, 0]
    ],
    [
      [0, 0, 0, 0],
      [0, 0, 1, 0],
      [1, 1, 1, 0],
      [0, 0, 0, 0]
    ],
    [
      [1, 1, 0, 0],
      [0, 1, 0, 0],
      [0, 1, 0, 0],
      [0, 0, 0, 0]
    ],
  ],
  [      // block2
    [
      [0, 0, 0, 0],
      [0, 1, 1, 0],
      [1, 1, 0, 0],
      [0, 0, 0, 0]
    ],
    [
      [1, 0, 0, 0],
      [1, 1, 0, 0],
      [0, 1, 0, 0],
      [0, 0, 0, 0]
    ],
    [
      [0, 0, 0, 0],
      [0, 1, 1, 0],
      [1, 1, 0, 0],
      [0, 0, 0, 0]
    ],
    [
      [1, 0, 0, 0],
      [1, 1, 0, 0],
      [0, 1, 0, 0],
      [0, 0, 0, 0]
    ],
  ],
  [      // block3
    [
      [1, 1, 1, 0],
      [0, 0, 1, 0],
      [0, 0, 0, 0],
      [0, 0, 0, 0]
    ],
    [
      [1, 1, 0, 0],
      [1, 0, 0, 0],
      [1, 0, 0, 0],
      [0, 0, 0, 0]
    ],
    [
      [0, 0, 0, 0],
      [1, 0, 0, 0],
      [1, 1, 1, 0],
      [0, 0, 0, 0]
    ],
    [
      [0, 1, 0, 0],
      [0, 1, 0, 0],
      [1, 1, 0, 0],
      [0, 0, 0, 0]
    ],
  ],
  [      // block4
    [
      [0, 0, 0, 0],
      [1, 1, 1, 1],
      [0, 0, 0, 0],
      [0, 0, 0, 0]
    ],
    [
      [0, 0, 1, 0],
      [0, 0, 1, 0],
      [0, 0, 1, 0],
      [0, 0, 1, 0]
    ],
    [
      [0, 0, 0, 0],
      [1, 1, 1, 1],
      [0, 0, 0, 0],
      [0, 0, 0, 0]
    ],
    [
      [0, 0, 1, 0],
      [0, 0, 1, 0],
      [0, 0, 1, 0],
      [0, 0, 1, 0]
    ],
  ],
  [      // block5
    [
      [0, 0, 0, 0],
      [0, 1, 1, 0],
      [0, 1, 1, 0],
      [0, 0, 0, 0]
    ],
    [
      [0, 0, 0, 0],
      [0, 1, 1, 0],
      [0, 1, 1, 0],
      [0, 0, 0, 0]
    ],
    [
      [0, 0, 0, 0],
      [0, 1, 1, 0],
      [0, 1, 1, 0],
      [0, 0, 0, 0]
    ],
    [
      [0, 0, 0, 0],
      [0, 1, 1, 0],
      [0, 1, 1, 0],
      [0, 0, 0, 0]
    ],
  ],
  [      // block6
    [
      [0, 0, 0, 0],
      [1, 1, 0, 0],
      [0, 1, 1, 0],
      [0, 0, 0, 0]
    ],
    [
      [0, 1, 0, 0],
      [1, 1, 0, 0],
      [1, 0, 0, 0],
      [0, 0, 0, 0]
    ],
    [
      [0, 0, 0, 0],
      [1, 1, 0, 0],
      [0, 1, 1, 0],
      [0, 0, 0, 0]
    ],
    [
      [0, 1, 0, 0],
      [1, 1, 0, 0],
      [1, 0, 0, 0],
      [0, 0, 0, 0]
    ],
  ]
];

// ブロックの色
const bcolor = ['#CC00CC', '#FFA500', '#CC0000',
                '#00CC00', '#CC0000', '#CCCC00', '#00CCCC'];


let col;  // blockのx座標 1..10
let row;  // blockのy座標 0..21

let syurui;     // ブロックの種類
let muki;       // ブロックの向き

let tensu = 0;   // 得点

let gameOn;     // ゲームが実行中である
let jikan;      // タイマー

// 画面の状態を記憶しておく
const joutai = new Array(23);

// 次のブロック
const nextBlock = { syurui: 0, muki: 0 };

function otoKaiten() {
  document.getElementById('kaiten').play();
}
function otoOchiru() {
  document.getElementById('ochiru').play();
}
function otoDon() {
  document.getElementById('don').play();
}
function otoGameover() {
    document.getElementById('gameover').play();
}
function otoKieru() {
    document.getElementById('kieru').play();
}
function otoZenbu() {
    document.getElementById('zenbu').play();
}

/**
 * 再描画処理
 * 画面データ(joutai配列)をもとにして再描画する。
 */
function reRender() {
    const cv = document.getElementById('game').getContext('2d');
    
    cv.clearRect(20, 0, 219, 419);
    // console.log('----------------------');
    for (y = 0; y < 21; y++) {
        for (x = 1; x < 11; x++) {
            if (joutai[y][x] !== 100) {
                // console.log('joutai y:' + y + ' x:' + x + ' j:' + joutai[y][x]);
                cv.fillStyle = bcolor[joutai[y][x]];
                cv.strokeStyle = '#333333';
                cv.lineWidth = 3;
                cv.fillRect(x * 20, y * 20, 20, 20);
                cv.strokeRect(x * 20, y * 20, 20, 20);
            }
        }
    }
}

/**
 * そろっている行を削除して上の行をつめる
 * 引数:
 *   y -- 行の番号
 */
function moveJoutai(y) {
  let x;
  
  if (y !== 0) {
    for (x = 1; x < 11; x++) {
      // 一つ上のセルの内容をコピーする
      joutai[y][x] = joutai[y - 1][x];
    }
    // 再帰処理。
    // すなわち、一つ上の行にも同じことをする。
    // ただ、第0行はしない。そこでこの再帰は終わる。
    moveJoutai(y - 1);
  }
  // 第0行は100で埋める。
  // なぜなら第0行は新しく生み出された行だから。
  for (x = 1; x < 11; x++) {
    joutai[0][x] = 100;
  }
}

/**
 * 得点処理
 * そろっている行が多いほど得点が上がる。
 */
function tokuten(num) {
  switch (num) {
  case 1:
    tensu = tensu + 10;
    otoKieru();
    break;
  case 2:
    tensu = tensu + 50;
    otoKieru();
    break;
  case 3:
    tensu = tensu + 100;
    otoKieru();
    break;
  case 4:
    tensu = tensu + 500;
    otoZenbu();
    break;
  }
  document.getElementById('tokuten').textContent = tensu;
}

/**
 * 横1列そろったら行う処理
 * そろっているかどうかを確認する。
 * そろってたら、そろっている行数分、得点する。
 * そろっている行を削除して再描画。
 */
function checkHorizontalLine() {
    const lineNo = [];   // 横一列そろっている行の配列
    let count;        // 100でないセルの数
    let x, y;

    // 1行ごとに100でないセルの数を数える
    for (y = 0; y < 21; y++) {
        count = 0;
        for (x = 1; x < 11; x ++) {
            if (joutai[y][x] !== 100) {
                count++;
            }
        }

        // もし、10個のセル全部が100でなかったら、
        if (count === 10) {
            lineNo.push(y);   // その行番号をlineNoに入れる。
        }
    }

    // lineNoには、そろっている行の番号が配列ではいっている。
    // 上の行から順番に、そろっている行を削除し、上の行を下につめる。
    if (lineNo.length > 0) {
        tokuten(lineNo.length);
        lineNo.forEach((ele) => {
            if (ele !== 0) {
                moveJoutai(ele);
            }
        });
        printJoutai();
        reRender();
    }
}


/**
 * kakunin -- 移動回転が可能かどうかを判定する
 * 引数
 *   col -- 現在のx座標(ブロックの左上)
 *   row -- 現在のy座標(ブロックの左上)
 *   syurui -- ブロックの種類(0...6)
 *   muki -- ブロックの向き(0...3)
 * 返り値
 *   true -- 移動は可能
 *   false -- 移動は不可能
 */
function kakunin(col, row, syurui, muki) {
  let x, y;
  let hantei = true;
  const thisBlock = block[syurui][muki];

  for (y = 0; y < 4; y++) {
	for (x = 0; x < 4; x++) {
      if (thisBlock[y][x] === 1) {
        // もしx座標が左の壁にはいったら 
        if (col + x < 1) {
          // x座標を戻す
		  col = col + 1;
		  return false;
		}	
        // もしx座標が右の壁に入ったら
        if (col + x > 11) {
          // x座標を戻す
		  col = col - 1;
		  return false;
        }
        // ブロックのどこかが 100 でなかったら
        // そこは移動可能な場所ではない。
		if (joutai[row + y][col + x] !== 100) {
		  return false;
		}
	  }
	}
  }
  return hantei;
}

/**
 * ブロックが底についたときの処理
 * 画面データの配列 joutai[][] の
 * 各セルにそのブロックのsyuruiを埋め込む。
 * syurui -- 0...6
 */
function atBottom(col, row, syurui, muki) {
  const thisBlock = block[syurui][muki];
  let x, y;

  for (y = 0; y < 4; y++) {
    for (x = 0; x < 4; x++) {
      if (thisBlock[y][x] === 1) {
        joutai[row + y][col + x] = syurui;
      }
    }
  }
}

/**
 * 次のブロックに処理を切り変える
 * 次のブロックの情報(種類、向き)は、
 * nextBlock(初期値) から得る。
 */
function changeToNextBlock() {
  col = Math.floor(Math.random() * 7) + 1;
  row = 0;
  syurui = nextBlock.syurui;
  muki = nextBlock.muki;

  // 新しいブロックが置けるかどうかを確認して、
  // もし置けなかったら、ゲームオーバー
  if (! kakunin(col, row, syurui, muki)) {
    otoGameover();
    alert('ゲームオーバー');
    gameOn = false;
  }
}

/**
 * 次のブロックの種類と向きを決定する。
 * 次のブロックを小窓に表示する。
 */
function makeNext() {
  const nextSyurui = Math.floor(Math.random() * 7);
  const nextMuki = Math.floor(Math.random() * 4);
  // 次のブロックを表示する小窓
  const nextGamen = document.getElementById('tsugi');
  const nextCV = nextGamen.getContext('2d');

  // 表示する前に消す
  nextCV.clearRect(0, 0, 79, 79);
  kaku(nextCV, 0, 0, nextSyurui, nextMuki);
  nextBlock.syurui = nextSyurui;
  nextBlock.muki = nextMuki;
}


/**
 * keyCode: left 37  up 38   right 39  down 40
 */
function ugokasu(e) {
  const gamegamen = document.getElementById('game');
  const cv = gamegamen.getContext('2d');	
  let mukiOrg;
  
  kesu(cv, col, row, syurui, muki);


  switch (e.keyCode) {
  case 37:
    // kakunin()がtrueならcolを左に移動
    if (kakunin(col - 1, row, syurui, muki)) {
      col = col - 1;
      otoKaiten();
    }
	break;
  case 38:
    mukiOrg = muki;
    muki = muki + 1;
    if (muki > 3) {
      muki = 0;
    }
    // もし、回転してkakunin()が false なら
    if (! kakunin(col, row, syurui, muki)) {
      // 向きを元に戻す
      muki = mukiOrg;
    }
	otoKaiten();
	break;
  case 39:
    // kakunin()が true なら colを右に移動
    if (kakunin(col + 1, row, syurui, muki)) {
	  col = col + 1;
      otoKaiten();
    }
    break;
  case 40:
    // kakunin()が true なら row を一つ下に移動
    if (kakunin(col, row + 1, syurui, muki)) {
      row = row + 1;
      otoOchiru();
    }
    // もし kakunin()が false なら、底についたということ
    else {
      kaku(cv, col, row, syurui, muki);
      otoDon();  // 底についた音
      // 画面データの配列 joutai にブロックの種類を埋め込む
      atBottom(col, row, syurui, muki);
      // そろっている行があれば、得点、削除、再描画を行う。
      checkHorizontalLine();
      printJoutai();
      changeToNextBlock();
      makeNext();
    }
	break;
  }

  kaku(cv, col, row, syurui, muki);

}

/**
 * ブロックを消す
 */
function kesu(cv, x, y, syurui, muki) {
  // 消す処理にする
  cv.globalCompositeOperation = 'destination-out';
  // 描く処理と同じだが、実際は消える。
  kaku(cv, x, y, syurui, muki);
  // 元にもどす
  cv.globalCompositeOperation = 'source-over';
}


/**
 * ブロックを描く
 * @param: cv -- canvas
 *         x  -- ブロックを描くx座標
 *         y  -- ブロックを描くy座標
 */
function kaku(cv, x, y, syurui, muki) {
  let i, t;

  cv.fillStyle = bcolor[syurui];
  cv.strokeStyle = '#333333';
  cv.lineWidth = 3;

  const thisBlock = block[syurui][muki];

  for (i = 0; i < 4; i++) {
	for (t = 0; t < 4; t++) {
	  if (thisBlock[i][t] === 1) {
		cv.fillRect((x + t) * 20, (y + i) * 20, 20, 20);
		cv.strokeRect((x + t) * 20, (y + i) * 20, 20, 20);
	  }
	}
  }
}

// 画面の各セルの状態を表示する  
function printJoutai() {
  let i, j;
  const joutaiArea = document.getElementById('joutai-area');
  let joutaiHtml;

  joutaiHtml = '<table>';
  for (i = 0; i < 23; i++) {
    joutaiHtml = joutaiHtml + `<tr><th>${i}:</th>`;
	for (j = 0; j < 12; j++) {
      joutaiHtml = `${joutaiHtml}<td>${joutai[i][j]}</td>`;
	}
    joutaiHtml = joutaiHtml + '</tr>';
  }
  joutaiHtml = joutaiHtml + '</table>';

  joutaiArea.innerHTML = joutaiHtml;
}

/**
 * 画面の各セルにデータを埋め込む
 * 100 -- 移動できるセル
 *  99 -- 壁
 *  0...6 -- ブロックが積まれているセル
 */
function setupJoutai() {
  let i, j;

  for (i = 0; i < 23; i++) {
	joutai[i] = new Array(12);
	for (j = 0; j < 12; j++) {
	  joutai[i][j] = 100;
	}
  }
  // 左壁
  for (i = 0; i < 23; i++) {
	joutai[i][0] = 99;
  }
  // 右壁
  for (i = 0; i < 23; i++) {
	joutai[i][11] = 99;
  }
  // 下壁
  for (j = 0; j < 12; j++) {
	joutai[21][j] = 99;
    joutai[22][j] = 99;
  }
}

/**
 * ユーザーが何もしなかったら、自動的に下向き矢印キーが
 * 押されたと同じことになる。しかも jikan はだんだんと
 * 短くなる。50ミリ秒以下になると1000ミリ秒に戻る。
 */
function jikandeugokasu() {
  if (gameOn) {
    // 実行中
    const ev = { keyCode: 40 };
    ugokasu(ev);

    jikan = jikan - 2;
    if (jikan < 50) {
      jikan = 1000;
    }
    document.getElementById('jikan').textContent = jikan;
    setTimeout(jikandeugokasu, jikan);
  }
}


function gamekaishi() {
  // Canvasを取得
  const gamegamen = document.getElementById('game');

  const cg =gamegamen.getContext('2d');

  // 画面を消す
  cg.clearRect(0, 0, 239, 439);

  // 得点をゼロにする
  tensu = 0;

  // ゲーム実行中にする
  gameOn = true;

  // タイマーを1秒にセット
  jikan = 1000;

  // 1秒ごとに jikandeugoku() を動かす。
  setTimeout( jikandeugokasu, jikan );

  // 画面データをつくる
  setupJoutai();
  // 画面データを表示
  printJoutai();
  
  col = 4;
  row = 0;

  syurui = Math.floor(Math.random() * 7);
  muki = Math.floor(Math.random() * 4);;

  kaku(cg, col, row, syurui, muki);

  makeNext();
}

function hajime() {

  // Canvasを取得
  const backgamen = document.getElementById('back');

  const cb =backgamen.getContext('2d');

  // 塗りを設定
  cb.fillStyle = '#CCCCCC';

  // 線を設定
  cb.strokeStyle = '#333333';
  cb.lineWidth = 3;

  // 左壁を描く
  let x = 0;
  let y = 0;
  let i;

  for (i = 0; i < 22; i++) {
	cb.fillRect(x, y, 20, 20);
	cb.strokeRect(x, y, 20, 20);
	y = y + 20;
  }
  // 右壁を描く
  x = 11 * 20;
  y = 0;
  for (i = 0; i < 22; i++) {
	cb.fillRect(x, y, 20, 20);
	cb.strokeRect(x, y, 20, 20);
	y = y + 20;
  }
  // 下壁を描く
  x = 20;
  y = 21 * 20;
  for (i = 0; i < 20; i++) {
	cb.fillRect(x, y, 20, 20);
	cb.strokeRect(x, y, 20, 20);
	x = x + 20;
  }
}
\end{lstlisting}




\end{document}

%% 修正時刻: Sat May  2 15:10:04 2020


%% 修正時刻: Mon Jun 22 20:40:40 2020
